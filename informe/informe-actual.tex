\documentclass[a4paper,12pt]{article}
\usepackage[utf8]{inputenc}
\usepackage[spanish]{babel}
\usepackage{geometry}
\geometry{margin=2.5cm}
\usepackage{graphicx}
\usepackage{amsmath, amssymb}
\usepackage{xcolor}
\usepackage{tikz, tcolorbox}
\usepackage{listings}
\usepackage{fancyvrb}
\usepackage{fancyhdr}
\usepackage{hyperref}
\usepackage{float}
\usepackage{eso-pic}

\title {\includegraphics[width=0.4\textwidth]{ing-uni.jpg}\\[2ex]{\textbf{Informe Proyecto\\ Restaurante v1.1}\\[1.5ex] Programacion II\\[3ex]{\normalsize\textbf{Profesor:} Guido Mellado}\\[1ex]{\normalsize\textbf{Ayudantes:} Fernando Valdéz y Joaquín Cantero}}\\[12ex]}
\author{Diego Hernandez, Benjamin Soto, Eduardo Necul}
\date{20 Noviembre de 2025}

\lstset{
    backgroundcolor=\color{gray!10},
    basicstyle=\ttfamily\footnotesize,
    frame=single,
    breaklines=true,
    keywordstyle=\color{blue},
    commentstyle=\color{green!50!black},
    stringstyle=\color{orange},
    showstringspaces=false
}

\begin{document}

\AddToShipoutPictureFG{% Anadir a la pagina actual
  \AtPageUpperLeft{% Punto de referencia: esquina superior izquierda
    \put(0.5cm, -2.2cm){% Mover x cm a la derecha y x cm hacia abajo
      \includegraphics[width=8cm]{departamento-inf.png}
    }
  }
}
{% Agrego esto para poner el logo del departamento de informatica en la zona superior.
}

\maketitle

\newpage
\tableofcontents
\newpage

\section{Introduccion}

\textit{Se nos hizo entrega de un codigo base para la funcion de un restaurante incompleto, en el cual nosotros debemos encargarnos de completarlo y refinarlo con la forma de programar \textbf{POO} (Programacion Orientada a Objetos), sin salirnos de las casillas del esqueleto inicial. Ademas nos preocuparemos de explicar cada cambio aplicado al codigo, que utilidad le dimos a las funciones vacias y el diagrama de clases en base a este, v1.0.} Se plantea el objetivo de crear una base de datos y sincronizarla en nuestro proyecto mediante \textbf{ORM} (Object Relational Mapping), ademas de generar un \textbf{CRUD} (Create, Read, Update, Delete) para cada entidad siguiendo el concepto de un ORM con la herramienta SQLAlchemy, nos enfocamos en la creacion de una interfaz grafica con un manejo claro y facil de utilizar con la libreria Customtkinter con la intencion en que los usuarios puedan interactuar con el sistema sin dificultad alguna. Se utilizara programacion funcional como las expresiones lambda, map, filter y reduce. Se implementaran graficos sencillos que utilicen estadisticas basadas en los datos del sistema y se creara la simulacion de un proceso de compra para generar boletas y registrar pedidos, v1.1.

\newpage

\section{Diagrama de Clases}

Para visualizar con claridad el problema presentado y su diseño \textbf{POO}, diseñamos un diagrama UML con el objetivo de visualizar las distintas clases utilizadas y sus interrelaciones a través de relaciones (asociación, agrupación, composición) y cardinalidades.

\begin{figure}[H]
    \centering
    \includegraphics[width=1.1\textwidth]{diagrama-restaurante.png}
    \caption{Diagrama de Clases de la aplicación.}.
\end{figure}

\end{document}
