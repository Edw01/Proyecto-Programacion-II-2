\documentclass[a4paper,12pt]{article}
\usepackage[utf8]{inputenc}
\usepackage[spanish]{babel}
\usepackage{geometry}
\geometry{margin=2.5cm}
\usepackage{graphicx}
\usepackage{amsmath, amssymb}
\usepackage{xcolor}
\usepackage{tikz, tcolorbox}
\usepackage{listings}
\usepackage{fancyvrb}
\usepackage{fancyhdr}
\usepackage{hyperref}
\usepackage{float}
\usepackage{eso-pic}

\title {\includegraphics[width=0.4\textwidth]{ing-uni.jpg}\\[2ex]{\textbf{Informe Proyecto\\ Restaurante v1.1}\\[1.5ex] Programacion II\\[3ex]{\normalsize\textbf{Profesor:} Guido Mellado}\\[1ex]{\normalsize\textbf{Ayudantes:} Fernando Valdéz y Joaquín Cantero}}\\[12ex]}
\author{Diego Hernandez, Benjamin Soto, Eduardo Necul}
\date{20 Noviembre de 2025}

\lstset{
    backgroundcolor=\color{gray!10},
    basicstyle=\ttfamily\footnotesize,
    frame=single,
    breaklines=true,
    keywordstyle=\color{blue},
    commentstyle=\color{green!50!black},
    stringstyle=\color{orange},
    showstringspaces=false
}

\begin{document}

\AddToShipoutPictureFG{% Anadir a la pagina actual
  \AtPageUpperLeft{% Punto de referencia: esquina superior izquierda
    \put(0.5cm, -2.2cm){% Mover x cm a la derecha y x cm hacia abajo
      \includegraphics[width=8cm]{departamento-inf.png}
    }
  }
}
{% Agrego esto para poner el logo del departamento de informatica en la zona superior.
}

\maketitle

\newpage
\tableofcontents
\newpage

\section{Introduccion}

\textit{Se nos hizo entrega de un codigo base para la funcion de un restaurante incompleto, en el cual nosotros debemos encargarnos de completarlo y refinarlo con la forma de programar \textbf{POO} (Programacion Orientada a Objetos), sin salirnos de las casillas del esqueleto inicial. Ademas nos preocuparemos de explicar cada cambio aplicado al codigo, que utilidad le dimos a las funciones vacias y el diagrama de clases en base a este, v1.0.} Se plantea el objetivo de crear una base de datos y sincronizarla en nuestro proyecto mediante \textbf{ORM} (Object Relational Mapping), ademas de generar un \textbf{CRUD} (Create, Read, Update, Delete) para cada entidad siguiendo el concepto de un ORM con la herramienta SQLAlchemy, nos enfocamos en la creacion de una interfaz grafica con un manejo claro y facil de utilizar con la libreria Customtkinter con la intencion en que los usuarios puedan interactuar con el sistema sin dificultad alguna. Se utilizara programacion funcional como las expresiones lambda, map, filter y reduce. Se implementaran graficos sencillos que utilicen estadisticas basadas en los datos del sistema y se creara la simulacion de un proceso de compra para generar boletas y registrar pedidos, v1.1.

\newpage

\section{Diagrama de Clases}

Para visualizar con claridad el problema presentado y su diseño \textbf{POO}, diseñamos un diagrama UML con el objetivo de visualizar las distintas clases utilizadas y sus interrelaciones a través de relaciones (asociación, agrupación, composición) y cardinalidades.

\begin{figure}[H]
    \centering
    \includegraphics[width=1.1\textwidth]{diagrama-restaurante.png}
    \caption{Diagrama de Clases de la aplicación.}.
\end{figure}

\newpage
\subsection{Explicacion del Diagrama}
La clase \textbf{\textit{AplicacionConPestanas}} es la principal clase del programa, ya que es ella la que se usa para configurar la interfaz de CustomTkinter e instanciar, relacionar y modificar el resto de clases del sistema, las cuales se describen a continuación brevemente:
\begin{itemize}

    \item La clase \textit{Stock} es aquella donde se administra la información de ingredientes, y que servirá para determinar la disponibilidad de menúes (o platos) en función de si existen los ingredientes requeridos o no, y tiene una relación de asociación con \textit{AplicacionConPestanas}, ya que pese a que esta última instancia un objeto Stock, no hay un contenedor de los objetos instanciados (lista) y ambas clases son independientes, en la que una referencia a otra.
    
    \item La clase \textit{Ingrediente} es aquella que configura la información de un ingrediente en base a sus tres atributos principales y que coinciden con el estándar de información del archivo CSV. Es el elemento principal de la clase Stock.
    
    \item La clase \textit{CrearMenu} es donde se configura la información de un menú/platillo y se configura en base a objetos Ingredientes, debido a la relación explicada en la clase Stock.
    \begin{itemize}
        \item[$\rightarrow$] Debido a la relación entre Stock y menúes, los objetos CrearMenu tienen una relación de agregación con los objetos Ingrediente, ya que toman ingredientes determinados y los configuran como parte de sus requisitos (los ingredientes requeridos o que forman parte de CrearMenu deben coincidir con ingredientes existentes en Stock).
        \item[$\rightarrow$] Exclusivamente, esta clase cuenta con una relación de abstracción + implementación con la clase IMenu, la cual corresponde a una interfaz que verifica que la construcción de CrearMenu cumpla los criterios solicitados.
    \end{itemize}
    
    \item La clase \textit{Pedido} es la clase donde se configura la información del pedido de un cliente en particular.
    
        \begin{itemize}
            \item[$\rightarrow$] La aplicación solo puede trabajar con máximo un Pedido a la vez, o con carencia de este, esto en una relación de asociación, ya que la existencia de ambos es bastante independiente (salvo por la instanciación que depende AplicacionConPestanas).
            \item[$\rightarrow$] La clase Pedido se configura en base a los menúes (CrearMenu) instanciados, y los va agregando en sí, teniendo la posibilidad de tener 1 o muchos menúes (de lo contrario Pedido estaría vacío y no habría datos con los que trabajar).
        \end{itemize}
        
    \item La clase \textit{BoletaFacade} tiene el proposito de generar la boleta de compra a partir de la recepcion de un objeto "pedido", pedido cual contiene lo comprado por el cliente en el restaurant.
    
    \item La clase \textit{CtkPDFViewer} es una clase auxiliar que sirve para crear la visualización del archivo PDF con la función de carga de PDF de \textit{AplicacionConPestanas}.
\end{itemize}

\newpage

\section{Diagrama Entidad-Relacion}

Para una mejor visualizacion del manejo de la base de datos, se creo un respectivo diagrama \textbf{MER} (Entidad-Relacion) con la funcion de ver sus respectivas cardinalidades, relaciones, entidades, atributos y propiedades, y mejorar la eficiencia en aplicar ORM, cambios, entre otros.

\begin{figure}[H]
    \centering
    \includegraphics[width=1.1\textwidth]{diagrama-mer.png}
    \caption{Diagrama MER de la DB.}.
\end{figure}

\subsection{Explicacion del Diagrama}

En este diagrama se nos presentan 4 entidades, cuyo nombre son: cliente, pedido, menu e ingrediente, se presenta una relacion cliente a pedido, donde UN cliente puede tener 0 o Muchos pedidos y donde los pedidos, 0 o Muchos, pueden estar compuestos de Uno o Muchos Menus, finalmente Menu posee una relacion con la entidad Ingrediente de, 0 o Muchos Menu estan asociado a Muchos ingredientes.\\

\textbf{Flujo:} Un cliente puede hacer un pedido o muchos pedidos, al hacer el pedido debera de seleccionar un menu para pedir o muchos si asi lo desea, al hacer un menu este requerira de ingredientes, los cuales como minimo deberan ser 2 o mas, es decir muchos ingrediente para preparar un respectivo menu.\\

\textbf{Nota:} Los rombos que se aprecian indican una tabla auxiliar, generada cuando existe una relacion Muchos a Muchos y la cual hace de intermediario.

\section{Explicacion abstracta del flujo del codigo}
Se abordara como indica el titulo, de manera abstracta el flujo del codigo, sin dar tantos detalles pero explicando lo suficiente para comprender en su totalidad el codigo hecho.

\end{document}
